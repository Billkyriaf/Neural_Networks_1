\subsection{Objective}

This assignment will implement the algorithms found in those libraries in C++ to the best ability of the author. The main focus
is to achieve good accuracy in a reasonable amount of run time. For this reason parallel programming techniques will be used.

\subsection{Equipment}

All of the test were run on a 64-bit Manjaro Linux system with an AMD Ryzen 7 5700G processor with 8 cores and 16 threads. The
processor has a base clock of 3.8 GHz and a boost clock of 4.6 GHz. The system memory is 16 GB DDR4 3200 MHz.

\subsection{Methodology}
\subsubsection{KNN}

The KNN algorithm is a supervised learning algorithm. The only parameters of the algorithm that is changed in this implementation
is the number of neighbors. The algorithm requires to calculate the distance between the test image and all the training images. The
distance is calculated using the Euclidean distance formula for each pixel of the image.

\subsubsection{K-means}

The K-means algorithm is an optimization of the KNN algorithm. Instead of calculating the distance between the test image and all The
training images, the K-means algorithm calculates the distance between the test image and the centroids of the clusters. The centroids
are the mean of the images in the cluster. This K-means implementation uses 10 clusters. The plain K-means algorithm has not parameters
to change.


\subsubsection{K-means with advanced clustering}

The next step in the optimization of the K-means algorithm is to use a more advanced clustering algorithm. The clustering algorithm used
in this implementation is the K-means++ algorithm. The K-means++ algorithm classifies the data into more clusters than the original ones.
With this optimization more features can be extracted from the data. The K-means++ algorithm has quite a few parameters to change such as
the number of clusters, the number of iterations and the portion of the data to use for the initial centroids. The exact implementations 
for all the algorithms will be explained in the next chapter.