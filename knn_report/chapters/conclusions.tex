The KNN algorithm has the greatest accuracy of the three algorithms. The disadvantage is that
the classification time for each image is very high. In the MNIST dataset this is not a big problem
because the training set is relatively small. Using bigger dataset would significantly increase the
classification time. The time complexity of the KNN algorithm is O\(m*n\) where n is the number of
training images and m is the number of test images.

The K-means algorithm has a better time complexity than the KNN algorithm because after the initial setup
the classification time is O\(10m\) where m is the number of test images. The disadvantage of the algorithm
is the lack of accuracy.

Finally the K-means++ algorithm has better accuracy than the K-means algorithm but still worse than the KNN.
Again after the initial setup the classification time is small. There are a few parameters that can be changed
to improve the accuracy of the algorithm. The number of clusters, the number of iterations and the portion of
the data to use for the initial centroids, and for the later fitting of them. 

In this projects the parameters are probably sub optimal but still produce an above average result. Further 
optimization of the parameters could improve the accuracy of the algorithm to levels directly comparable to
the KNN algorithm.   